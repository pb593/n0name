% Copyright 2004 by Till Tantau <tantau@users.sourceforge.net>.
%
% In principle, this file can be redistributed and/or modified under
% the terms of the GNU Public License, version 2.
%
% However, this file is supposed to be a template to be modified
% for your own needs. For this reason, if you use this file as a
% template and not specifically distribute it as part of a another
% package/program, I grant the extra permission to freely copy and
% modify this file as you see fit and even to delete this copyright
% notice. 

\documentclass{beamer}

\setbeamercovered{transparent}
% There are many different themes available for Beamer. A comprehensive
% list with examples is given here:
% http://deic.uab.es/~iblanes/beamer_gallery/index_by_theme.html
% You can uncomment the themes below if you would like to use a different
% one:
%\usetheme{AnnArbor}
%\usetheme{Antibes}
%\usetheme{Bergen}
%\usetheme{Berkeley}
%\usetheme{Berlin}
%\usetheme{Boadilla}
%\usetheme{boxes}
%\usetheme{CambridgeUS}
%\usetheme{Copenhagen}
%\usetheme{Darmstadt}
%\usetheme{default}
%\usetheme{Frankfurt}
%\usetheme{Goettingen}
%\usetheme{Hannover}
%\usetheme{Ilmenau}
%\usetheme{JuanLesPins}
%\usetheme{Luebeck}
\usetheme{Madrid}
%\usetheme{Malmoe}
%\usetheme{Marburg}
%\usetheme{Montpellier}
%\usetheme{PaloAlto}
%\usetheme{Pittsburgh}
%\usetheme{Rochester}
%\usetheme{Singapore}
%\usetheme{Szeged}
%\usetheme{Warsaw}

\title{Conversation Security using KleeQ}

% A subtitle is optional and this may be deleted
\subtitle{Part II Project}

\author{Pavel Berkovich}
% - Give the names in the same order as the appear in the paper.
% - Use the \inst{?} command only if the authors have different
%   affiliation.

\institute[University of Cambridge] % (optional, but mostly needed)
{
  \inst{}%
  University of Cambridge \\ 
  \vspace{1cm}
  \inst{}%
  \Small{Supervised by Dr.~Richard Clayton}
}
% - Use the \inst command only if there are several affiliations.
% - Keep it simple, no one is interested in your street address.

\date{February 4, 2016}
% - Either use conference name or its abbreviation.
% - Not really informative to the audience, more for people (including
%   yourself) who are reading the slides online

%\subject{Theoretical Computer Science}
% This is only inserted into the PDF information catalog. Can be left
% out. 

% If you have a file called "university-logo-filename.xxx", where xxx
% is a graphic format that can be processed by latex or pdflatex,
% resp., then you can add a logo as follows:

\pgfdeclareimage[height=0.5cm]{university-logo}{logo.png}
\logo{\pgfuseimage{university-logo}}

% Delete this, if you do not want the table of contents to pop up at
% the beginning of each subsection:
\AtBeginSubsection[]
{
  \begin{frame}<beamer>{Outline}
    \tableofcontents[currentsection,currentsubsection]
  \end{frame}
}

% Let's get started
\begin{document}

\begin{frame}
  \titlepage
\end{frame}

\begin{frame}{Outline}
  \tableofcontents
  % You might wish to add the option [pausesections]
\end{frame}

% Section and subsections will appear in the presentation overview
% and table of contents.
\section{Project and its goals}
\begin{frame}{Problems of P2P secure communication}
    \begin{block}{Problem 1: Contact Discovery}<1>
        How do we know where to send our messages?
    \end{block}
    \begin{block}{Problem 2: Trust Establishment}<1>
        How do we know our peers are who they say they are?
    \end{block}
    \begin{block}{Problem 3: Conversation Security}<1-2>
        How do we encrypt the messages, what data do we attach to them, and what security protocols do we perform?
    \end{block}
    \begin{block}{Problem 4: Transport Privacy}<1>
        What is the mechanics for actually sending the message so as to hide the message metadata (sender, recepient, time, size etc)?
    \end{block}
\end{frame}

\begin{frame}{\textit{KleeQ (Readon et al, 2006)}}
    \begin{itemize}
        \item conversation security protocol for P2P \textit{ad-hoc} group communication
        \item security properties:
            \begin{itemize}
                \item confidentiality of message content
                \item message integrity
                \item forward and backward secrecy
                \item message authorship repudiation
                \item conversation participation repudiation
                \item anonymity preserving
            \end{itemize}
        \item very hacky and unstable implementation in Python
    \end{itemize}
\end{frame}

\begin{frame}{Goals of the project}{Brief reminder}
    \begin{block}{Goal 1: Implementation}
        Implement the protocol in Java. See how it performs, test scalability limits.
    \end{block}
    \begin{block}{Goal 2: Messenger Prototype}
        Build a simple prototype of a messenger to show that the protocol works.
    \end{block}
\end{frame}

\section{Accomplishments}
\begin{frame}{Accomplishments}
    \begin{block}{Architecture}
            \begin{itemize}
                \item Asynchronous communicaiton with callbacks
                \item Inheritance hierarchy of message types
            \end{itemize}
    \end{block}
    \begin{block}{Some protocol components}
        \begin{itemize}
            \item Group establishment
            \item Derivation of common secret + encryption/decryption
        \end{itemize}
    \end{block}
    \begin{block}{Interface}
        A simple CLI interface, for testing.
    \end{block}
    \begin{block}{Secondary Components}
        \begin{itemize}
            \item Online contact discovery ("address book")
            \item Store-n-forward service
        \end{itemize}
    \end{block}
\end{frame}


\section{To-Do's}
\begin{frame}{To-Do's}
    \begin{block}{Patching Algorithm}
        An somewhat unusual algorithm for message exchange suggested by paper.
        Currently done by pseudo-multicast.
    \end{block}
    \begin{block}{Transcript Verification}
        Procedure for verifying the global transcript, specified in the paper.
        No integrity check at the moment.
    \end{block}
    \begin{block}{Improved key mangement}
        Independent recomputation of common secret based on the results of transcript
        verification. Gives forward/backward secrecy.
    \end{block}


\end{frame}

\section{Challenges Encountered}
\begin{frame}{Challenges Encountered}
    \pause
    \begin{block}{Challenge 1: Protocol gaps}
        The original paper omits \textit{a lot} of detail. Have to re-design some
        parts independently. 
    \end{block}
    \pause
    \begin{block}{Challenge 2: Phase order problem}
        To test a conversation security protocol, have to write a lot of "scaffolding" first. This needs to be done \textit{before}, not after writing the protocol.
    \end{block}
    \pause
    \begin{block}{Challenge 3: Private IP addresses}
        P2P is made complicated by most hosts not having public IP addresses.
        Had to write a simple store-and-forward service.
    \end{block}
\end{frame}


\section{Planning \& Timing}
\begin{frame}{Timing}
    \begin{tabular*}{0.9\textwidth}{c | l}
        Date & Milestone \\ 
        \hline
        14-02-2016 & Patching Algorithm implemented \\
        21-02-2016 & Transcript integrity verification done \\ 
        28-02-2016 & Key rotation implemented \\
        06-03-2016 & Clean up and bundle into a usable library\\
        21-04-2016 & Dissertation written up \\
    \end{tabular*}
\end{frame}

\section{Q\&A}
\begin{frame}{Q\&A}
    \Large{Do you have any questions?}
\end{frame}

\begin{frame}
    \centering
    \Large{Thank you!}
\end{frame}

\end{document}



\documentclass[a4paper, 12pt]{article}

\newcommand\litem[1]{\item{\bfseries #1}}

\begin{document}
\title{Group conversation security using KleeQ}
\author{Pavel Berkovich, St John's College}
\maketitle

\section{Introduction and Description of the Work}
\label{intro}
The recent public outcry in response to mass surveillance programs carried out by governments around the world led to creation of multiple messaging systems which emphasised security of communication. Some have been more successful than others, but each of them has been found to have some flaws or deficiencies. Broadly speaking, there are four problems that each security-oriented messenger needs to solve \cite{sok}:
\begin{description}
    \item[Problem 1: Contact Discovery] \hfill \\
        How do we find out at which IP addresses our peers currently reside? How do we know where to send our messages?
    \item[Problem 2: Trust Establishment]\hfill \\
        Once we know where our peers are, how do we know that they are who they say they are? How do we make sure that they are not being impersonated by a malicious adversary?
    \item[Problem 3: Conversation Security]\hfill \\
        Once we are sure that we are talking to the right parties, how do we protect the security and privacy of the messages' content? In other words, how do we encrypt the messages, what data do we attach to them, and what security protocols do we perform?
    \item[Problem 4: Transport Privacy]\hfill \\
        What is the mechanics for actually sending the message so as to hide the message metadata (e.g. sender identity, recipient identity, conversation to which the message belongs etc).
\end{description}

\vspace{\baselineskip}
\noindent
Multiple protocols, with different threat definitions and varying levels of security, have been developed for each of these tasks. The aim of this project is to explore and build on KleeQ \cite{kleeq}, one of the protocols aimed at providing conversation security (Problem 3 in the list above). Given a group of trusted parties residing at known addresses, it gives the following security guarantees for their conversation \cite{sok}:
\begin{itemize}
    \item confidentiality of message content
    \item message integrity
    \item forward secrecy
    \item backward secrecy
    \item message authorship repudiation
    \item conversation participation repudiation
\end{itemize}
KleeQ has been designed for devices with transient connectivity (e.g. communication via Bluetooth or wireless), but the ideas introduced by this protocol can be used in the general setting of group messaging over a network.

\vspace{\baselineskip}
\noindent
In this project, KleeQ will be completely re-implemented into a general-purpose network conversation security protocol, preserving the security properties mentioned above. The performance characteristics will be tested, and the use of the protocol will be demonstrated by constructing a simple demo messaging app.

\vspace{\baselineskip}
\noindent
Optionally, if time allows, an attempt shall be made to make use of the most recent and secure third-party solutions to solve Problems 1, 2 and 4 from the list above, thereby producing a full-blown secure messaging application.


\section{Resources Required}
All the development will be based on the resources provided by my own machine and the PWF. In addition, it is expected that a commercial cloud hosting service (DigitalOcean, Heroku or similar) will be used, for the purpose of hosting the server side of the application. The backup strategy involves periodically pushing the code to a private GitHub repository and keeping a hand-written project log. No other resources will be required.


\section{Starting Point}
As of the starting date of the project, the following relevant background:
\begin{itemize}
    \item Experience of programming in Java, at the level of University courses \emph{Programming in Java} and \emph{Further Java}.
    \item Understanding of cryptographic primitives and computer security fundamentals, as taught in Part IB \emph{Security I} course.
    \item Limited experience of writing server-side code in Python using Flask.
\end{itemize}
\vspace{\baselineskip}


\noindent
The successful completion of the core part of the project will require the following:
\begin{itemize}
    \item Learning about the fundamentals of conversation security, understanding what security properties it entails and what the common attack strategies are.
    \item Understanding the algorithms and data structures introduced by KleeQ \cite{kleeq}.
    \item Learning to write security-oriented code in Java.
\end{itemize}

\noindent
To complete the optional part, the following will need to be done:
\begin{itemize}
    \item Understanding the most common attack strategies employed by adversaries against secure messengers.
    \item Reading research papers and technical documentation describing the APIs presented by the third-party libraries in use.
\end{itemize}


\section{Substance and Structure of the Project}
\subsection{Core Part}
The core part of the project will involve re-implementing KleeQ, to make it a universal conversation security protocol. Initially, it will be necessary to obtain a deep understanding of how KleeQ operates which will be done by reading the relevant research material \cite{kleeq}, as well as studying the source code of the current implementation. Then, the parts of the protocol which require re-designing to allow network communication (as opposed to ad-hoc communication via Bluetooth or similar) will be identified, and the necessary design decisions will be made. The next step will be implementing the protocol in Java and testing it locally. At that point, additional checks will be made to ensure that the new implementation retains all the security properties of the original one and, if necessary, eliminate the possible loopholes in the code.

\vspace{\baselineskip}
\noindent
The next step will be turning the result of the above into a simple desktop P2P messaging application. It is important to understand that producing a full-blown secure messaging system is outside the scope of the core part of this project -- at this stage, the aim will be to write some simple and not necessarily secure "scaffolding" code to produce a prototype, for the purposes of demonstrating the work of the new protocol implementation. In particular, it is expected that a minimalistic graphical user interface and a simple server-based contact discovery component will be implemented at this point.


\subsection{Optional Extensions}
If time allows, the most recent and secure approaches may be used to convert the aforementioned demo application into a proper secure messaging system. As mentioned in Section \ref{intro}, this would require solving the problems of secure contact discovery, trust establishment and transport privacy. These problems can be solved in the following ways:
\begin{itemize}
    \item DHT with query anonymisation for contact discovery
    \item self-auditable public key logs for trust establishment (using CONIKS \cite{coniks})
    \item Tor-based hidden service for transport privacy and metadata hiding
\end{itemize}
All of the above are available in the form of open-source libraries which can be used in the project without major modifications.

\vspace{\baselineskip}
\noindent
In addition, an effort may be made to improve the usability of the application. With this in mind, the GUI will be refined to match the usability of the most ubiquitous messenger applications (e.g. WhatsApp, Viber, Telegram etc).




\section{Success Criteria}
There shall be two measures of success for this project:
\begin{enumerate}
    \item Implementation of a working messaging system which would be possible to use.
    \item Preservation of the security properties of the original KleeQ implementation, namely:
        \begin{itemize}
            \item confidentiality of message content
            \item message integrity
            \item forward secrecy
            \item backward secrecy
            \item message authorship repudiation
            \item conversation participation repudiation
        \end{itemize}
\end{enumerate}




\section{Timetable and Milestones}
It is important to structure the project in a way which would allow to evenly distribute the workload over the available time and minimise risks of missing deadlines. In addition, as pointed out by previous Part II students, it would be beneficial to finish the dissertation write-up at least two weeks in advance of the official deadline, to allow more time for Tripos revision. With this in mind, the following schedule has been set:


\subsection*{Weeks 1-2 (Oct 25 -- Nov 6)}
Do preliminary reading and understand the algorithms and data structures used by KleeQ, referring to the existing implementation as necessary. Identify the parts of the protocol which require modification, and design the object-oriented structure of the new implementation. Decide on which standard Java classes will need to be used. Find out what common implementation mistakes result is security loopholes. Arrange the necessary infrastructure, such as a back-up repository and cloud hosting space.

\vspace{0.7\baselineskip}
\noindent
\textit{Milestone:} The necessary knowledge of the protocol acquired, design and infrastructure ready -- can begin writing code.

\subsection*{Weeks 3-7 (Nov 7 -- Dec 11)}
Implement the protocol in Java. Test locally. Watch out for most common implementation vulnerabilities. Make sure the original security guarantees are in place.

\vspace{0.7\baselineskip}
\noindent
\textit{Milestone:} The conversation security protocol implemented and tested.

\subsection*{Weeks 9-10 (Dec 12 -- Dec 25)}
Turn the protocol implementation into a library which would be usable by third parties in their applications.

\subsection*{Week 11 (Dec 26 -- Jan 1)}
A week-long holiday is planned for these dates. No work will be done during this time.

\subsection*{Weeks 12-13 (Jan 2 -- Jan 15)}
Implement the "address book" server application to allow contact discovery. Make sure the server has the most current information about user's status. Test the practicality of KleeQ's conversation concepts.

\subsection*{Week 14 (Jan 16 -- Jan 22)}
Write-up the progress report and submit it one week ahead of the deadline.

\subsection*{Weeks 15-16 (Jan 23 -- Feb 5)}
Create a minimalistic GUI. Make sure that the cases of asynchrony (device going offline or coming back online) are adequately handled.

\vspace{0.7\baselineskip}
\noindent
\textit{Milestone:} An operational messaging application is ready. Core part of the project finished, the success criteria satisfied.


\subsection*{Weeks 17-19 (Feb 6 -- Feb 26)}
Evaluate how the protocol performs over a network under high load, and try to tune it to perform better. Refine the GUI of the prototype to allow usage by non-experts.

\vspace{0.7\baselineskip}
\noindent
\textit{Milestone:} The application looks good and the performance is satisfactory.

\subsection*{Weeks 20-21 (Feb 27 -- Mar 11)}
Implement the optional extensions as time allows. Make sure that none of the previously implemented security guarantees are broken.

\subsection*{Week 22 (Mar 12 -- Mar 18)}
A week-long holiday is planned for these dates. No work will be done during this time.


%\subsection*{Weeks 11-12 (Jan 4 -- Jan 17)}
%Use Orchid, the Tor library for Java, to implement metadata hiding. Make sure the application is still usable and the introduced delays are tolerable.

%\vspace{0.7\baselineskip}
%\noindent
%\textit{Milestone:} Transport security provided -- traffic anonymised, some other metadata hidden.

%\subsection*{Weeks 13-15 (Jan 18 -- Feb 7)}
%Use CONIKS self-auditable public key logs to do trust establishment -- integrate a CONIKS client into the messenger and run a CONIKS server on the cloud hosting platform. Test the performance of the system, and measure how much traffic is required for log auditing.

%\vspace{0.7\baselineskip}
%\noindent
%\textit{Milestone:} Trust established securely. MitM attacks facilitated by the trust authority can be detected.

%\subsection*{Weeks 16-18 (Feb 8 -- Feb 28)}
%Use a DHT with anonymous queries to implement anonymous contact discovery. Measure the performance of the DHT, see how much time is necessary for synchronisation.

%\vspace{0.7\baselineskip}
%\noindent
%\textit{Milestone:} Contact discovery anonymised. The previous server-based "address book" replaced.

\subsection*{Weeks 23-27 (Mar 19 -- Apr 22)}
Write up the dissertation. Use the project log to recollect the details of what work has been done and how. Pay special attention to the Evaluation section, describing in detail the previously made performance measurements. Arrange regular meetings with the supervisor to receive feedback and iteratively refine the work. This part of work will take a long time, since it will be interleaved with Tripos revision over the Easter Vacation.

\vspace{0.7\baselineskip}
\noindent
\textit{Milestone:} Dissertation largely written. The supervisor and the overseers are satisfied with the content.

\subsection*{Weeks 28-29 (Apr 23 -- May 13)}
No project work is scheduled for these days -- the intention is to use this time for Tripos revision. It also provides a "safety buffer" in case more time is required, for one reason or another. If necessary, final adjustments to the text of the dissertation will be made at this time.

\vspace{0.7\baselineskip}
\noindent
\textit{Milestone:} Dissertation printed, bound and submitted at least two weeks ahead of the deadline.

\begin{thebibliography}{9}

\bibitem{coniks}
  M. Melara et al,
  \emph{CONIKS: Bringing Key Transparency to End Users},
  Princeton University, Stanford University/Electronic Frontier Foundation,
  2014.

\bibitem{sok}
    N. Unger et al,
    \emph{SoK: Secure Messaging},
    University of Waterloo, University of Bonn, Stanford University, Electronic Frontier Foundation, Fraunhofer FKIE,
    2015.
    
\bibitem{kleeq}
  J. Readon et al,
  \emph{KleeQ: Asynchronous Key Management for Dynamic Ad-Hoc Networks},
  University of Waterloo, Maplesoft,
  2007.

\end{thebibliography}

\end{document}
